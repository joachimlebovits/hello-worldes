%\documentclass[letterpaper]{article}
\documentclass [10pt,a4paper] {article}

\usepackage[utf8]{inputenc}
\usepackage[T1]{fontenc}
\usepackage{chapterbib}
\usepackage{bibentry}
\usepackage{epsf}
\usepackage{amsmath, amsthm}
\usepackage{amsfonts}
% \usepackage{amssymb}
\usepackage{graphicx,epsfig,float}
\usepackage{lscape}
\usepackage[usenames]{color}
\usepackage{pifont}
\usepackage{yhmath}
\usepackage{amssymb,amsmath,amsfonts}
\usepackage{tabularx}
\usepackage{amsthm}
\usepackage{dsfont}
\usepackage{epsfig}
\usepackage{mathrsfs}
\usepackage{fullpage}
\usepackage{eurosym}

\newcommand{\ds}{\displaystyle}
\newcommand{\fin}{\Box}
\newcommand{\saut}[1]{\hfill\\[#1]}
\newcommand{\difrac}{\displaystyle \frac}
\newcommand{\dist}{\textrm{dist}}
\newcommand{\mbf}{\textbf}
\newcommand{\cl}{\centerline}
\newcommand{\is}{\backsimeq}
\newcommand{\bit} {\begin{itemize} }
\newcommand{\eit} {\end{itemize} }
\newcommand\E{\mbox{\sf E}}
\newcommand\Q{\mbox{\sf Q}}
\newcommand\ho{H\"{o}lder }
\newcommand{\loc}{\mathrm{loc}}
\newcommand{\Log}{\mathrm{Log} \,}
\newcommand{\diam}{\mathrm{diam}}
\newcommand{\cqfd}{\quad \Box}
\newcommand{\Abar}{\overline{A}}
\newcommand{\ftild}{\widetilde{f}}
\newcommand{\Ci}{\mathrm{Ci}}
\newcommand{\enstq}[2]{\left\{#1~\middle|~#2\right\}}
\newcommand{\tim}{t_{i-1}}
\newcommand{\ti}{t_{i}}
\newcommand{\tjm}{t_{j-1}}
\newcommand{\tj}{t_{j}}
\newcommand{\CF}{{\cal{F}}}
\newcommand{\CFm}{{\cal{F}}_{\tim}}
\newcommand{\CFjm}{{\cal{F}}_{\tjm}}
\newcommand{\La}{$L^2_{ad}([a,b], \Omega)$}
\newcommand{\Lae}{$L^2_{ad}([a,b], \Omega)$ }
\newcommand{\Ela}{${\cal{L}}_{ad}(\Omega,L^2[a,b])$}
\newcommand{\Elae}{${\cal{L}}_{ad}(\Omega,L^2[a,b])$ }
\newcommand{\VQM}{ < \hspace{-.4em} M \hspace{-.5em} >}
\newcommand{\dVQM}{ d \hspace{-.4em} < \hspace{-.4em} M \hspace{-.5em} >}
\newcommand{\sgn}{\mbox{sgn} }


\newcommand{\bN}{\mathbb{N}}
\newcommand{\Z}{\mathbb{Z}}
%\newcommand{\Q}{\mathbb{Q}}
\newcommand{\R}{\mathbb{R}}
\newcommand{\C}{\mathbb{C}}
\newcommand{\Rbar}{\overline{\mathbb{R}}}
\newcommand{\Pro}{\mathbb{P}}
%\newcommand{\E}{\mathbb{E}}
\newcommand{\Var}{\mathrm{Var}}
\newcommand{\Cov}{\mathrm{Cov}}
\newcommand{\T}{\mathbb{T}}
\newcommand{\D}{\mathbb{D}}
\newcommand{\INDIK}{1 \! \! 1}

\newcommand{\vsp}{\vspace{2ex}}
\newcommand{\Vsp}{\vspace{2em}}
\newcommand{\hs}{\hspace{0.1em}}
%\newcommand{\h1}{\hspace{0.1cm}}


\newcommand{\one}{\ifmmode {\sf 1}\hspace{-.26em}{\sf
l}\hspace{-.35em}{\sf \_} \else ${\sf 1}\hspace{-.26em}{\sf
l}\hspace{-.35em}{\sf \_}$ \fi}


\def\ben{\begin{enumerate}}
\def\een{\end{enumerate}}

\def\iti{\item[(i)]}
\def\biti{\item[\bfseries(i)]}

\def\itii{\item[(ii)]}
\def\bitii{\item[\bfseries(ii)]}

\def\itiii{\item[(iii)]}
\def\bitiii{\item[\bfseries(iii)]}

\def\itiv{\item[(iv)]}
\def\bitiv{\item[\bfseries(iv)]}

\def\itv{\item[(v)]}
\def\bitv{\item[\bfseries(v)]}

\def\itvi{\item[(vi)]}
\def\bitvi{\item[\bfseries(vi)]}

\def\ita{\item[(a)]}
\def\bita{\item[\bfseries(a)]}

\def\itb{\item[(b)]}
\def\bitb{\item[\bfseries(b)]}

\def\itc{\item[(c)]}
\def\bitc{\item[\bfseries(c)]}

\def\itd{\item[(d)]}
\def\bitd{\item[\bfseries(d)]}

%%LETTRES Caligraphiques
\def\cA{{\cal A}}
\def\cB{{\cal B}}
%\def\cC{{\cal C}}
\def\cC{C}
\def\cD{{\cal D}}
\def\cE{{\cal E}}
\def\cF{{\cal F}}
\def\cG{{\cal G}}
\def\cH{{\cal H}}
\def\cI{{\cal I}}
\def\cJ{{\cal J}}
\def\cK{{\cal K}}
\def\cL{{\cal L}}
\def\cM{{\cal M}}
\def\cN{{\cal N}}
\def\cO{{\cal O}}
\def\cP{{\cal P}}
\def\cQ{{\cal Q}}
\def\cR{{\cal R}}
\def\cS{{\cal S}}
\def\cU{{\cal T}}
\def\cU{{\cal U}}
\def\cV{{\cal V}}
\def\cW{{\cal W}}
\def\cX{{\cal X}}
\def\cY{{\cal Y}}
\def\cZ{{\cal Z}}
\def\cnu{{\cal \nu}}


%%LETTRES Anglaises
\newcommand{\sA}{\mathscr{A}}
\newcommand{\sB}{\mathscr{B}}
\newcommand{\sC}{\mathscr{C}}
\newcommand{\sD}{\mathscr{D}}
\newcommand{\sE}{\mathscr{E}}
\newcommand{\sF}{\mathscr{F}}
\newcommand{\sG}{\mathscr{G}}
\newcommand{\sH}{\mathscr{H}}
\newcommand{\sI}{\mathscr{I}}
\newcommand{\sJ}{\mathscr{J}}
\newcommand{\sK}{\mathscr{K}}
\newcommand{\sL}{\mathscr{L}}
\newcommand{\sM}{\mathscr{M}}
\newcommand{\SN}{\mathscr{N}}
\newcommand{\sO}{\mathscr{O}}
\newcommand{\sP}{\mathscr{P}}
\newcommand{\sQ}{\mathscr{Q}}
\newcommand{\sR}{\mathscr{R}}
\newcommand{\sS}{\mathscr{S}}
\newcommand{\sT}{\mathscr{T}}
\newcommand{\sU}{\mathscr{U}}
\newcommand{\sV}{\mathscr{V}}
\newcommand{\sW}{\mathscr{W}}
\newcommand{\sX}{\mathscr{X}}
\newcommand{\sY}{\mathscr{Y}}
\newcommand{\sZ}{\mathscr{Z}}


\def\cAC{{\cal AC}}

\def\bo{{\tiny{ $\square$}}}


%%%%%%%%%%%%%%%%%%%%%%%%%%%%%%%%%%%%%%%%%
%%LETTRES EN GRAS
\def\bA{{\mathbb A}}
\def\bB{{\mathbb B}}
\def\bC{{\mathbb C}}
\def\bD{{\mathbb D}}
\def\bE{{\mathbb E}}
\def\bF{{\mathbb F}}
\def\bL{{\mathbb L}}
\def\bN{{\mathbb N}}
\def\bP{{\mathbb P}}
\def\bQ{{\mathbb Q}}
\def\bR{{\mathbb R}}
\def\bZ{{\mathbb Z}}

\def\i1{\mathds{1}} %% fonction indicatrice

\def\si{\text{sign}} %% fonction sign
\def\sp{\vspace{1cm}}
\def\2sp{\vspace{2cm}}
\def\3sp{\vspace{3cm}}
\def\[ent{[\hskip -1.5pt [}
\def\]ent{]\hskip -1.5pt ]}
\def\rent{{\bf ]}\hskip -1.5pt {\bf ]}}
\def\lent{{\bf [}\hskip -1.5pt {\bf [}}


\def\sp{\vspace{1cm}}
\def\2sp{\vspace{2cm}}
\def\3sp{\vspace{3cm}}


%%%% debut macro %%%%
\newenvironment{changemargin}[2]{\begin{list}{}{%
\setlength{\topsep}{0pt}%
\setlength{\leftmargin}{0pt}%
\setlength{\rightmargin}{0pt}%
\setlength{\listparindent}{\parindent}%
\setlength{\itemindent}{\parindent}%
\setlength{\parsep}{0pt plus 1pt}%
\addtolength{\leftmargin}{#1}%
\addtolength{\rightmargin}{#2}%
}\item }{\end{list}}
%%%% fin macro %%%%

\def\h1{\hspace{0.1cm}}
\def\bbbr{{\rm {\bf R}}} % Real numbers
\def\bbbn{{\rm {\bf N}}} % Natural numbers
\def\bbbz{{\rm {\bf Z}}}
\def\bbbc{{\rm {\bf C}}} % complex numbers
\newcommand{\reals}{\ifmmode {\sf I}\hspace{-.15em}{\sf R} \else ${\sf
I}\hspace{-.15em}{\sf R}$ \fi}
\newcommand\osc{\mbox{osc} }
\newcommand\varep{\varepsilon}
%\newcommand\h1{\hspace{0.1cm}}
\renewcommand{\Box}{\mbox{\rule{1ex}{1ex}}}
\renewcommand{\leq}{\leqslant}
\renewcommand{\geq}{\geqslant}

\newcommand{\sifbm}{\mathbf{B}}


\def\gg{{\textquotedblleft}}
\def\dd{{\textquotedblright}}
\def\h1{{\hspace{0.1cm}}}

\newtheorem{theo}{Theorem}[section]
\newtheorem{theodef}{Theorem-Definition}[section]
\newtheorem{defi}{Definition}[section]
\newtheorem{prop}[theo]{Proposition}
\newtheorem{note}{Note}[section]
\newtheorem{proper}{Propertie}%[section]
\newtheorem{propers}{Properties}[section]
\newtheorem{lem}[theo]{Lemma}%[section]
\newtheorem{ex}[theo]{Example}%[section]
\newtheorem{cor}[theo]{Corollary}%[section]
\newtheorem{exa}[theo]{Example}%[section]
\newtheorem{conj}[theo]{Conjecture}%[section]
\newtheorem{rem}[theo]{Remark}
\newtheorem{exo}{Exercice}
%\newtheorem{Pro}[theo]{Proposition}

\newenvironment{defappli}[4]{\begin{array}{cccl} %
#1 \, : & #2 & \rightarrow & #3 \\ & #4 & \mapsto &}%
{\end{array}}

\newenvironment{g}[2]{\begin{array}{cl} %
<\hspace{-0.2cm}<\hspace{-0.1cm}#1,\ \hspace{-0.45cm}
\hspace{-1cm}&#2\hspace{-0.1cm}>\hspace{-0.2cm}>}%
{\end{array}}

\newenvironment{defappliab}[6]{\begin{array}{cccccl} %
#1 \, : & #2    & \stackrel{\unboldmath{M_{h}}}{\rightarrow}   & #3  &
\stackrel{\unboldmath{\zeta}}{\rightarrow}     & #4 \\
        & #5    &  \mapsto      &  #6   & \mapsto        &  }%
{\end{array}}

\def\qed {{% set up
\parfillskip=0pt % so \par doesnt push \square to left
\widowpenalty=10000 % so we dont break the page before \square
\displaywidowpenalty=10000 % ditto
\finalhyphendemerits=0 % TeXbook exercise 14.32
%
% horizontal
\leavevmode % \nobreak means lines not pages
\unskip % remove previous space or glue
\nobreak % don't break lines
\hfil % ragged right if we spill over
\penalty50 % discouragement to do so
\hskip.2em % ensure some space
\null % anchor following \hfill
\hfill % push \square to right
$\square$% % the end-of-proof mark
%
% vertical
\par}} % build paragraph

\newenvironment{prz}{{\bfseries \textup{Proof.}}}{}
\newenvironment{pr}{\begin{proof}[\bfseries \textup{Proof.}]}{\end{proof}}
\newenvironment{praa}{{\bfseries \textup{Proof }}}{}
\newtheorem{proe}{Proof}
\renewcommand{\theproe}{}
\newenvironment{pree}{\begin{proe}}
 { \end{proe}}
\newtheorem{pro}{Proof}
\renewcommand{\thepro}{}
\newenvironment{pre}{\begin{Pro}}
 { $\qed$ \end{Pro}}

\def\mathtitre#1{
\font\tenrm=cmr10 scaled \magstep#1
\font\sevenrm=cmr7 scaled \magstep#1
\font\fiverm=cmr5 scaled \magstep#1
\font\teni=cmmi10 scaled \magstep#1
\font\seveni=cmmi7 scaled \magstep#1
\font\fivei=cmmi5 scaled \magstep#1
\font\tensy=cmsy10 scaled \magstep#1
\font\sevensy=cmsy7 scaled \magstep#1
\font\fivesy=cmsy5 scaled \magstep#1
\font\tenex=cmex10 scaled \magstep#1
\textfont0=\tenrm \scriptfont0=\sevenrm \scriptscriptfont0=\fiverm
\textfont1=\teni \scriptfont1=\seveni \scriptscriptfont1=\fivei
\textfont2=\tensy \scriptfont2=\sevensy \scriptscriptfont2=\fivesy
\textfont3=\tenex \scriptfont3=\tenex \scriptscriptfont3=\tenex
}

\makeatletter
\renewcommand\theequation{\thesection.\arabic{equation}}
\@addtoreset{equation}{section}
\makeatother

\def\independent{{\perp\!\!\!\!\perp}}
%\newcommand\independent{\protect\mathpalette{\protect\independenT}{\perp}}
\def\independenT#1#2{\mathrel{\rlap{$#1#2$}\mkern2mu{#1#2}}}

\setlength{\parindent}{0em}

% keywords
\def\keywordname{{\bf Keywords:}}
\newcommand{\keywords}[1]{
\par\addvspace\baselineskip\noindent\keywordname\enspace\ignorespaces#1}

%\renewcommand{\thefootnoteremember}{\Alph{footnote}}

\renewcommand{\thefootnote}{\Alph{footnote}}

\newcommand{\footnoteremember}[2]{ 
 \footnote{#2}
 \newcounter{#1}
 \setcounter{#1}{\value{footnote}}
 \addtocounter{#1}{0}
}
\newcommand{\footnoterecall}[1]{
 \footnotemark[\value{#1}]
}
\begin{document}
\setlength{\parindent}{0em}
\makeatletter
\renewcommand\theequation{\thesection.\arabic{equation}}
\@addtoreset{equation}{section}
\makeatother


\begin{flushleft}
{ \bfseries IUT de Saint-Denis}			\hspace{9.25cm}				Année $2016$-$2017$

{ \bfseries Remise à niveau	}					

{ \bfseries Feuille de T.D $n^{0}1$}		 \hspace{9.5cm} Joachim Lebovits
\end{flushleft}



%
%\title{Answers to the Associate Editor on his comments on \\ \textit{Stochastic Integration with respect to Multifractional Brownian motion via tangent Fractional Brownian motions}}
%\author{Joachim \textsc{Lebovits} \and Jacques
%\textsc{Lévy Véhel} \and Erick  \textsc{Herbin}}
\date\today
%\maketitle



\vspace{1cm}



\section{Fractions \& quotients}

\begin{exo}[Calculs]

\textcolor{white}{s}


Les fractions suivantes sont-elles égales?
\bit
\item $\frac{3}{8}$ et $\frac{36}{96}$; $\frac{74}{20}$ et $\frac{19}{5}$.
\item $\frac{28}{64}$ et $\frac{3}{8}$; $\frac{35}{63}$ et $\frac{20}{36}$.
\eit
\end{exo}
\vspace{0.25cm}

\begin{exo}[Calculs suite]

\textcolor{white}{s}

Effectuer sous forme fractionnaire les calculs suivants.
\bit
\item $A = \frac{17}{5}+\frac{7}{5}$; $B= \frac{1}{4}+\frac{1}{2}$.
\item $C = \frac{3}{5}+\frac{3}{10}$; $D= \frac{8}{15}+\frac{2}{3}$.
\item $E = \frac{5}{6}+\frac{2}{5}$; $F= 8+\frac{3}{6}$.
\item $G = \frac{3}{14}+\frac{1}{6}$;
\eit
\end{exo}
\vspace{0.25cm}



\begin{exo}[Calculs suite \& fin]

\textcolor{white}{s}

Effectuer sous forme fractionnaire les calculs suivants.
\bit
\item $A = \frac{2}{3} \times 4$; $B= \frac{1}{5}\times\frac{3}{2}$.
\item $C = \frac{8}{5}\times\frac{2}{3}$; $D= \frac{4}{3}\div 2$.
\item $E = \frac{5}{4}\div\frac{1}{2}$; $F= \frac{3}{52}\div 5$.
\item $G = \frac{8}{3}\div\frac{9}{5}$; $H = 12\div\frac{5}{3}$
\item $I = \frac{A\times B}{C+D}$; $J = \frac{I}{A}\times (C+D)$
\eit
\end{exo}
\vspace{0.25cm}




\section{Proportionnalité}

\vspace{0.25cm}


\begin{exo}[Répartition proportionnelle]
\textcolor{white}{s}

Trois villes de la Somme se partagent une indemnité de $630 000$ \euro, proportionnellement à leur nombre d’habitants. \hspace{-0.2cm} Abbeville a une populations de $35000$ habitants, Flixecourt en compte $5 000$ et Mareuil-Caubert $2 000$.


\ben
\item Quelle est l’indemnité reçue par chaque ville ?
\item Quelle est l’indemnité par habitant ?
\een
\end{exo}

\vspace{0.25cm}

\begin{exo}[Vitesse \& débit]
\textcolor{white}{s}

\ben
\item Un avion traverse l’atlantique de Paris à Fort-de-France en $7$h$45$. La
distance entre les deux villes est de $6800$ km. Quelle est sa vitesse en km/h puis
en miles par heure (à $0.1$ près)?

\item Une baignoire d'une contenance de $0.4$ $m^{3}$ se remplit avec un débit de $0.75$ l/s. Au bout de
combien de temps sera-t-elle remplie aux $3/4$.
\een
Rappel: un mile nautique correspond à $1852$ mètres.
\end{exo}

\vspace{0.25cm}

\begin{exo}[Echelle]
\textcolor{white}{s}

\ben
\item Sur une carte IGN au $1/25 000$ ème, Antoine mesure une distance de $12,5$cm
entre les deux étapes de sa randonnée. Quelle est la longueur de sa randonnée en km ?
\item  Gauthier doit faire une randonnée à vélo de $75$ km. Il souhaite vérifier sur
une carte routière au $1/50 000$ ème cette distance. Combien de cm doit-il
obtenir ?
\item Un architecte à construit une maquette d’un gymnase dont les mesures
réelles sont $70$ m et $42$ m. Sa maquette a pour dimension $52,5$ cm et $31,5$ cm.
Pourra-t-il retrouver l’échelle utilisée ?
\een
\end{exo}

\vspace{0.25cm}

\begin{exo}[Partage inversement proportionnel]
\textcolor{white}{s}

La commune de Candé prévoit dans son budget $25 000$ \euro au titre des activités
culturelles et sociales. La répartition doit s'effectuer ainsi:

- 10\% sont attribués à la Coopérative scolaire,\\
- 24\% du reste sont affectés à la Caisse d'Entraide,\\
- le reste est distribué entre 3 clubs de loisirs inversement
proportionnellement aux nombres d'années d'activités de chacun soit
respectivement 3 ans, 2 ans et 1 an.\\

Effectuer le détail de cette répartition.
\end{exo}

\vspace{0.25cm}

\begin{exo}[Proportionnalité \& pourcentage]

\textcolor{white}{s}

Un paquet de $800$g de céréales est vendu $2,8$ \euro.
\ben
\item Quel sera le prix d'un paquet de $1,5$ kg si les prix sont proportionnels?
\item Le même paquet est vendu avec $20\%$ de produit gratuit en plus au prix de $6,4$ \euro. Est-ce une offre intéressante?
\item Un autre paquet de $300$g, dont $50$g gratuit, est vendu au prix de $0,9$ \euro.
La promotion est-elle honnête ?
\een
\end{exo}

\vspace{0.25cm}


\section{Pourcentages}

\begin{exo}[pourcentages]

\textcolor{white}{s}

Dans un sondage effectué auprès de $1 200$ personnes, $0,75$ \% des personnes
ne se sont pas prononcées. 
A quel nombre de personnes ce pourcentage correspond-il?
\end{exo}
\vspace{0.25cm}


\begin{exo}[pourcentages: les bases]

\textcolor{white}{s}
\ben
\item Après une baisse de $20$\%, le prix d'un article est affiché à $280$ \euro. Quel est
son prix initial?
\item  Un article coûte $32$\euro. Les soldes sont prévues dans 8 jours et on peut
espérer une baisse de $20$ à $50$\% du prix. Quelles seront les bornes entre lesquelles les économies seront comprises si on a le ``courage'' d’attendre?
\item  Un article passe d’un prix initial de $22,5$\euro \ à un prix final de $270$\euro.\ Quel est
le pourcentage d’augmentation?
\item Un article a un prix final de $1207,50$\euro \ après une augmentation de $15$\%.
Quel est son prix initial?
\een
\end{exo}
\vspace{0.25cm}


\begin{exo}[Augmentations et baisses successives]

\textcolor{white}{s}

Le litre d’essence, après avoir augmenté de $55$\% au mois de mai, a baissé de $28$\% au mois de septembre.
\ben
\item Sachant qu’il coûtait $1$\euro\ au mois d' avril, quel est sont coût au mois d'octobre ?
\item  Quel est le coefficient multiplicateur ? En déduire le pourcentage global
d’augmentation.
\een
\end{exo}
\vspace{0.25cm}


\begin{exo}[Pourcentage de pourcentage]

\textcolor{white}{s}

Au mois d'août $2000$, le ministère du travail a publié l’évolution du chômage au
cours des douze derniers mois. La baisse du nombre de chômeurs en un an est de
$15,4$\%. Ce nombre s’établit fin août 2000 à $2\ 328\ 800$ demandeurs d’emploi.
Le taux de chômage dans la population active (la population qui travaille ou
recherche du travail) s’élève à $9,6$\%.

\ben
\item Déterminer l’effectif total de la population active.
\item Si le chômage continue de décroître au même rythme, combien y-aura-t-il
de chômeurs en moins dans un an ?
\item En considérant que la population active reste stable, quel sera le taux de
chômage dans la population active ? En déduire la baisse de ce taux en un an.
\een
\end{exo}
\vspace{0.25cm}





\end{document}





