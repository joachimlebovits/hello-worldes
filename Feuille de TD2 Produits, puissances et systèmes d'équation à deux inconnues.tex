%\documentclass[letterpaper]{article}
\documentclass [10pt,a4paper] {article}

\usepackage[utf8]{inputenc}
\usepackage[T1]{fontenc}
\usepackage{chapterbib}
\usepackage{bibentry}
\usepackage{epsf}
\usepackage{amsmath, amsthm}
\usepackage{amsfonts}
% \usepackage{amssymb}
\usepackage{graphicx,epsfig,float}
\usepackage{lscape}
\usepackage[usenames]{color}
\usepackage{pifont}
\usepackage{yhmath}
\usepackage{amssymb,amsmath,amsfonts}
\usepackage{tabularx}
\usepackage{amsthm}
\usepackage{dsfont}
\usepackage{epsfig}
\usepackage{mathrsfs}
\usepackage{fullpage}
\usepackage{eurosym}
\usepackage{hyperref}


\newcommand{\ds}{\displaystyle}
\newcommand{\fin}{\Box}
\newcommand{\saut}[1]{\hfill\\[#1]}
\newcommand{\difrac}{\displaystyle \frac}
\newcommand{\dist}{\textrm{dist}}
\newcommand{\mbf}{\textbf}
\newcommand{\cl}{\centerline}
\newcommand{\is}{\backsimeq}
\newcommand{\bit} {\begin{itemize} }
\newcommand{\eit} {\end{itemize} }
\newcommand\E{\mbox{\sf E}}
\newcommand\Q{\mbox{\sf Q}}
\newcommand\ho{H\"{o}lder }
\newcommand{\loc}{\mathrm{loc}}
\newcommand{\Log}{\mathrm{Log} \,}
\newcommand{\diam}{\mathrm{diam}}
\newcommand{\cqfd}{\quad \Box}
\newcommand{\Abar}{\overline{A}}
\newcommand{\ftild}{\widetilde{f}}
\newcommand{\Ci}{\mathrm{Ci}}
\newcommand{\enstq}[2]{\left\{#1~\middle|~#2\right\}}
\newcommand{\tim}{t_{i-1}}
\newcommand{\ti}{t_{i}}
\newcommand{\tjm}{t_{j-1}}
\newcommand{\tj}{t_{j}}
\newcommand{\CF}{{\cal{F}}}
\newcommand{\CFm}{{\cal{F}}_{\tim}}
\newcommand{\CFjm}{{\cal{F}}_{\tjm}}
\newcommand{\La}{$L^2_{ad}([a,b], \Omega)$}
\newcommand{\Lae}{$L^2_{ad}([a,b], \Omega)$ }
\newcommand{\Ela}{${\cal{L}}_{ad}(\Omega,L^2[a,b])$}
\newcommand{\Elae}{${\cal{L}}_{ad}(\Omega,L^2[a,b])$ }
\newcommand{\VQM}{ < \hspace{-.4em} M \hspace{-.5em} >}
\newcommand{\dVQM}{ d \hspace{-.4em} < \hspace{-.4em} M \hspace{-.5em} >}
\newcommand{\sgn}{\mbox{sgn} }


\newcommand{\bN}{\mathbb{N}}
\newcommand{\Z}{\mathbb{Z}}
%\newcommand{\Q}{\mathbb{Q}}
\newcommand{\R}{\mathbb{R}}
\newcommand{\C}{\mathbb{C}}
\newcommand{\Rbar}{\overline{\mathbb{R}}}
\newcommand{\Pro}{\mathbb{P}}
%\newcommand{\E}{\mathbb{E}}
\newcommand{\Var}{\mathrm{Var}}
\newcommand{\Cov}{\mathrm{Cov}}
\newcommand{\T}{\mathbb{T}}
\newcommand{\D}{\mathbb{D}}
\newcommand{\INDIK}{1 \! \! 1}

\newcommand{\vsp}{\vspace{2ex}}
\newcommand{\Vsp}{\vspace{2em}}
\newcommand{\hs}{\hspace{0.1em}}
%\newcommand{\h1}{\hspace{0.1cm}}


\newcommand{\one}{\ifmmode {\sf 1}\hspace{-.26em}{\sf
l}\hspace{-.35em}{\sf \_} \else ${\sf 1}\hspace{-.26em}{\sf
l}\hspace{-.35em}{\sf \_}$ \fi}


\def\ben{\begin{enumerate}}
\def\een{\end{enumerate}}

\def\iti{\item[(i)]}
\def\biti{\item[\bfseries(i)]}

\def\itii{\item[(ii)]}
\def\bitii{\item[\bfseries(ii)]}

\def\itiii{\item[(iii)]}
\def\bitiii{\item[\bfseries(iii)]}

\def\itiv{\item[(iv)]}
\def\bitiv{\item[\bfseries(iv)]}

\def\itv{\item[(v)]}
\def\bitv{\item[\bfseries(v)]}

\def\itvi{\item[(vi)]}
\def\bitvi{\item[\bfseries(vi)]}

\def\ita{\item[(a)]}
\def\bita{\item[\bfseries(a)]}

\def\itb{\item[(b)]}
\def\bitb{\item[\bfseries(b)]}

\def\itc{\item[(c)]}
\def\bitc{\item[\bfseries(c)]}

\def\itd{\item[(d)]}
\def\bitd{\item[\bfseries(d)]}

%%LETTRES Caligraphiques
\def\cA{{\cal A}}
\def\cB{{\cal B}}
%\def\cC{{\cal C}}
\def\cC{C}
\def\cD{{\cal D}}
\def\cE{{\cal E}}
\def\cF{{\cal F}}
\def\cG{{\cal G}}
\def\cH{{\cal H}}
\def\cI{{\cal I}}
\def\cJ{{\cal J}}
\def\cK{{\cal K}}
\def\cL{{\cal L}}
\def\cM{{\cal M}}
\def\cN{{\cal N}}
\def\cO{{\cal O}}
\def\cP{{\cal P}}
\def\cQ{{\cal Q}}
\def\cR{{\cal R}}
\def\cS{{\cal S}}
\def\cU{{\cal T}}
\def\cU{{\cal U}}
\def\cV{{\cal V}}
\def\cW{{\cal W}}
\def\cX{{\cal X}}
\def\cY{{\cal Y}}
\def\cZ{{\cal Z}}
\def\cnu{{\cal \nu}}


%%LETTRES Anglaises
\newcommand{\sA}{\mathscr{A}}
\newcommand{\sB}{\mathscr{B}}
\newcommand{\sC}{\mathscr{C}}
\newcommand{\sD}{\mathscr{D}}
\newcommand{\sE}{\mathscr{E}}
\newcommand{\sF}{\mathscr{F}}
\newcommand{\sG}{\mathscr{G}}
\newcommand{\sH}{\mathscr{H}}
\newcommand{\sI}{\mathscr{I}}
\newcommand{\sJ}{\mathscr{J}}
\newcommand{\sK}{\mathscr{K}}
\newcommand{\sL}{\mathscr{L}}
\newcommand{\sM}{\mathscr{M}}
\newcommand{\SN}{\mathscr{N}}
\newcommand{\sO}{\mathscr{O}}
\newcommand{\sP}{\mathscr{P}}
\newcommand{\sQ}{\mathscr{Q}}
\newcommand{\sR}{\mathscr{R}}
\newcommand{\sS}{\mathscr{S}}
\newcommand{\sT}{\mathscr{T}}
\newcommand{\sU}{\mathscr{U}}
\newcommand{\sV}{\mathscr{V}}
\newcommand{\sW}{\mathscr{W}}
\newcommand{\sX}{\mathscr{X}}
\newcommand{\sY}{\mathscr{Y}}
\newcommand{\sZ}{\mathscr{Z}}


\def\cAC{{\cal AC}}

\def\bo{{\tiny{ $\square$}}}


%%%%%%%%%%%%%%%%%%%%%%%%%%%%%%%%%%%%%%%%%
%%LETTRES EN GRAS
\def\bA{{\mathbb A}}
\def\bB{{\mathbb B}}
\def\bC{{\mathbb C}}
\def\bD{{\mathbb D}}
\def\bE{{\mathbb E}}
\def\bF{{\mathbb F}}
\def\bL{{\mathbb L}}
\def\bN{{\mathbb N}}
\def\bP{{\mathbb P}}
\def\bQ{{\mathbb Q}}
\def\bR{{\mathbb R}}
\def\bZ{{\mathbb Z}}

\def\i1{\mathds{1}} %% fonction indicatrice

\def\si{\text{sign}} %% fonction sign
\def\sp{\vspace{1cm}}
\def\2sp{\vspace{2cm}}
\def\3sp{\vspace{3cm}}
\def\[ent{[\hskip -1.5pt [}
\def\]ent{]\hskip -1.5pt ]}
\def\rent{{\bf ]}\hskip -1.5pt {\bf ]}}
\def\lent{{\bf [}\hskip -1.5pt {\bf [}}


\def\sp{\vspace{1cm}}
\def\2sp{\vspace{2cm}}
\def\3sp{\vspace{3cm}}


%%%% debut macro %%%%
\newenvironment{changemargin}[2]{\begin{list}{}{%
\setlength{\topsep}{0pt}%
\setlength{\leftmargin}{0pt}%
\setlength{\rightmargin}{0pt}%
\setlength{\listparindent}{\parindent}%
\setlength{\itemindent}{\parindent}%
\setlength{\parsep}{0pt plus 1pt}%
\addtolength{\leftmargin}{#1}%
\addtolength{\rightmargin}{#2}%
}\item }{\end{list}}
%%%% fin macro %%%%

\def\h1{\hspace{0.1cm}}
\def\bbbr{{\rm {\bf R}}} % Real numbers
\def\bbbn{{\rm {\bf N}}} % Natural numbers
\def\bbbz{{\rm {\bf Z}}}
\def\bbbc{{\rm {\bf C}}} % complex numbers
\newcommand{\reals}{\ifmmode {\sf I}\hspace{-.15em}{\sf R} \else ${\sf
I}\hspace{-.15em}{\sf R}$ \fi}
\newcommand\osc{\mbox{osc} }
\newcommand\varep{\varepsilon}
%\newcommand\h1{\hspace{0.1cm}}
\renewcommand{\Box}{\mbox{\rule{1ex}{1ex}}}
\renewcommand{\leq}{\leqslant}
\renewcommand{\geq}{\geqslant}

\newcommand{\sifbm}{\mathbf{B}}


\def\gg{{\textquotedblleft}}
\def\dd{{\textquotedblright}}
\def\h1{{\hspace{0.1cm}}}

\newtheorem{theo}{Theorem}[section]
\newtheorem{theodef}{Theorem-Definition}[section]
\newtheorem{defi}{Definition}[section]
\newtheorem{prop}[theo]{Proposition}
\newtheorem{note}{Note}[section]
\newtheorem{proper}{Propertie}%[section]
\newtheorem{propers}{Properties}[section]
\newtheorem{lem}[theo]{Lemma}%[section]
\newtheorem{ex}[theo]{Example}%[section]
\newtheorem{cor}[theo]{Corollary}%[section]
\newtheorem{exa}[theo]{Example}%[section]
\newtheorem{conj}[theo]{Conjecture}%[section]
\newtheorem{rem}[theo]{Remark}
\newtheorem{exo}{Exercice}
%\newtheorem{Pro}[theo]{Proposition}

\newenvironment{defappli}[4]{\begin{array}{cccl} %
#1 \, : & #2 & \rightarrow & #3 \\ & #4 & \mapsto &}%
{\end{array}}

\newenvironment{g}[2]{\begin{array}{cl} %
<\hspace{-0.2cm}<\hspace{-0.1cm}#1,\ \hspace{-0.45cm}
\hspace{-1cm}&#2\hspace{-0.1cm}>\hspace{-0.2cm}>}%
{\end{array}}

\newenvironment{defappliab}[6]{\begin{array}{cccccl} %
#1 \, : & #2    & \stackrel{\unboldmath{M_{h}}}{\rightarrow}   & #3  &
\stackrel{\unboldmath{\zeta}}{\rightarrow}     & #4 \\
        & #5    &  \mapsto      &  #6   & \mapsto        &  }%
{\end{array}}

\def\qed {{% set up
\parfillskip=0pt % so \par doesnt push \square to left
\widowpenalty=10000 % so we dont break the page before \square
\displaywidowpenalty=10000 % ditto
\finalhyphendemerits=0 % TeXbook exercise 14.32
%
% horizontal
\leavevmode % \nobreak means lines not pages
\unskip % remove previous space or glue
\nobreak % don't break lines
\hfil % ragged right if we spill over
\penalty50 % discouragement to do so
\hskip.2em % ensure some space
\null % anchor following \hfill
\hfill % push \square to right
$\square$% % the end-of-proof mark
%
% vertical
\par}} % build paragraph

\newenvironment{prz}{{\bfseries \textup{Proof.}}}{}
\newenvironment{pr}{\begin{proof}[\bfseries \textup{Proof.}]}{\end{proof}}
\newenvironment{praa}{{\bfseries \textup{Proof }}}{}
\newtheorem{proe}{Proof}
\renewcommand{\theproe}{}
\newenvironment{pree}{\begin{proe}}
 { \end{proe}}
\newtheorem{pro}{Proof}
\renewcommand{\thepro}{}
\newenvironment{pre}{\begin{Pro}}
 { $\qed$ \end{Pro}}

\def\mathtitre#1{
\font\tenrm=cmr10 scaled \magstep#1
\font\sevenrm=cmr7 scaled \magstep#1
\font\fiverm=cmr5 scaled \magstep#1
\font\teni=cmmi10 scaled \magstep#1
\font\seveni=cmmi7 scaled \magstep#1
\font\fivei=cmmi5 scaled \magstep#1
\font\tensy=cmsy10 scaled \magstep#1
\font\sevensy=cmsy7 scaled \magstep#1
\font\fivesy=cmsy5 scaled \magstep#1
\font\tenex=cmex10 scaled \magstep#1
\textfont0=\tenrm \scriptfont0=\sevenrm \scriptscriptfont0=\fiverm
\textfont1=\teni \scriptfont1=\seveni \scriptscriptfont1=\fivei
\textfont2=\tensy \scriptfont2=\sevensy \scriptscriptfont2=\fivesy
\textfont3=\tenex \scriptfont3=\tenex \scriptscriptfont3=\tenex
}

\makeatletter
\renewcommand\theequation{\thesection.\arabic{equation}}
\@addtoreset{equation}{section}
\makeatother

\def\independent{{\perp\!\!\!\!\perp}}
%\newcommand\independent{\protect\mathpalette{\protect\independenT}{\perp}}
\def\independenT#1#2{\mathrel{\rlap{$#1#2$}\mkern2mu{#1#2}}}

\setlength{\parindent}{0em}

% keywords
\def\keywordname{{\bf Keywords:}}
\newcommand{\keywords}[1]{
\par\addvspace\baselineskip\noindent\keywordname\enspace\ignorespaces#1}

%\renewcommand{\thefootnoteremember}{\Alph{footnote}}

\renewcommand{\thefootnote}{\Alph{footnote}}

\newcommand{\footnoteremember}[2]{ 
 \footnote{#2}
 \newcounter{#1}
 \setcounter{#1}{\value{footnote}}
 \addtocounter{#1}{0}
}
\newcommand{\footnoterecall}[1]{
 \footnotemark[\value{#1}]
}
\begin{document}
\setlength{\parindent}{0em}
\makeatletter
\renewcommand\theequation{\thesection.\arabic{equation}}
\@addtoreset{equation}{section}
\makeatother


\begin{flushleft}
{ \bfseries IUT de Saint-Denis}			\hspace{9.25cm}				Année $2016$-$2017$

{ \bfseries Remise à niveau	}					

{ \bfseries Feuille de T.D $n^{0}2$}		 \hspace{9.5cm} Joachim Lebovits
\end{flushleft}




\date\today




\vspace{1cm}


\section{Produits}

\begin{exo}%[s]

\textcolor{white}{s}

Factoriser les expressions suivantes.
\begin{align*}
 &A(x):= x^{2}+2x+1;&  &B(x):=x^{2}-6x+1;& 
 &C(x):=x^{2}-9&\\
&D(x):= 4x^{2}-12x+9;& &E(x):= 16-4x^{2}.&
\end{align*}
\end{exo}


\begin{exo}%[Calculs suite]


\begin{enumerate}
\item Développer les expressions suivantes.
\begin{align*}
 &A(x):= (2x+6)^{2};&  &B(x):= (4-3x)^{2};& 
 &C(x):=(2x+1)(2x-1)&\\
&D(x):= (3+2x)^{2};& &E(x):= (x+1/2)(x-1/2).&
\end{align*}
\item Résoudre l'équation suivante:
\begin{equation}
 \label{eirufheriuh}
 2x+1=0
\end{equation}
En utilisant la ou les solutions de l'équation \eqref{eirufheriuh}, déterminer l'ensemble des solutions de l'équation suivante
\begin{equation*}
 4x^{2}-1=0
\end{equation*}
 
\item Démontrer les égalités suivantes.
\begin{align*}
&(2y-5)^{2}-(3+2y)^{2}=-8(4y-2);& 
&\&& 
&(2x-1)^{2}-(3+x)^{2}=(3x+2)(x-4).&
 \end{align*}
\end{enumerate}

\end{exo}
\vspace{0.25cm}


%
%\begin{exo}[Calculs suite \& fin]
%
%\textcolor{white}{s}
%
%Effectuer sous forme fractionnaire les calculs suivants.
%\bit
%\item $A = \frac{2}{3} \times 4$; $B= \frac{1}{5}\times\frac{3}{2}$.
%\item $C = \frac{8}{5}\times\frac{2}{3}$; $D= \frac{4}{3}\div 2$.
%\item $E = \frac{5}{4}\div\frac{1}{2}$; $F= \frac{3}{52}\div 5$.
%\item $G = \frac{8}{3}\div\frac{9}{5}$; $H = 12\div\frac{5}{3}$
%\item $I = \frac{A\times B}{C+D}$; $J = \frac{I}{A}\times (C+D)$
%\eit
%\end{exo}
%\vspace{0.25cm}




\section{Puissances}

\vspace{0.25cm}


\begin{exo}%[Répartition proportionnelle]
\textcolor{white}{s}

Compléter les calculs suivants.

\begin{eqnarray*}
&& a^n\times a^p=...;~~~~~\frac{a^n}{a^p}=...;~~~~~(a^n)^p=...\\
&& a^n\times a^{-p}=...;~~~~~\frac{a^{-n}}{a^p}=...;~~~~~(a^n)^{\frac{1}{p}}=...
\end{eqnarray*}
Application:
\begin{eqnarray*}
&&6^3\times 6^4=...;~~~~~\frac{4^{22}}{4^{23}}=...;~~~~~(2^3)^4=...\\
&& 6^7\times 3^{-14}=...;~~~~~\frac{2^{-20}}{4^{-10}}=...;~~~~~(7^{24})^{\frac{1}{6}}=...
\end{eqnarray*}

\begin{eqnarray*}
&& 5^3\times ...=5^5;~~~~~\frac{3^6}{...}=3^{-10};~~~~~(11^8)^..=11^2\\
&& 12^3\times ...=\frac{1}{12};~~~~~\frac{6^{-2}}{...}=6;~~~~~(9^2)^{...}=3
\end{eqnarray*}

\begin{eqnarray*}
&& (-2)^3=...;~~~~~\frac{(-3)^6}{...}=3^{-10};~~~~~((-11)^8)^..=11^2\\
&& (-12)^3\times ...=\frac{-1}{12};~~~~~\frac{(-6)^{-2}}{...}=6;~~~~~((-5)^2)^{...}=25.
\end{eqnarray*}
\end{exo}

\vspace{0.25cm}

\begin{exo}%[Vitesse \& débit]
\textcolor{white}{s}

Calculer les expressions suivantes et donner l'écriture 
 du résultat en notation scientifique.\\
 
 \begin{center}
$A:=${\Large $ \frac{50\times 10^{5}\times 0,6 \times 10^{9}}{60 \times {(10^{5})}^{5}}$; } \hspace{0.25cm} $B:=${\Large $ \frac{54\times 10^{4}\times 12 \times 10^{9}}{0,36 \times {(10^{-10})}^{3}}$;} \hspace{0.25cm}  $C:=${\Large $\frac{2^{-11}\times 3^{4}\times 12^{7} \times 6^{9}}{72^{4} \times {(4^{2})}^{3}}$.}
\end{center}



%
%
%\begin{center}
%$
%\begin{tabular}{lll}
%$
%\begin{array}{lc}
%A:=  \frac{50\times 10^{5}\times 0,6 \times 10^{9}}{60 \times {(10^{5})}^{5}}.   %.&
%\end{array}$& 
%$
%\begin{array}{lc}
%B:=\frac{54\times 10^{4}\times 12 \times 10^{9}}{0,36 \times {(10^{-10})}^{3}}.
%\end{array}$& 
%$
%\begin{array}{lc}
%C:=\frac{2^{-11}\times 3^{4}\times 12^{7} \times 6^{9}}{72^{4} \times {(4^{2})}^{3}}.
%\end{array}$
%% \\
%%   2.1 & 2.2 & 2.3 \\
%\end{tabular}
%$
%\end{center}

\end{exo}



\vspace{0.25cm}


%========== Troisième partie ===============================

\section{Systèmes d'équations à deux inconnues}


\begin{exo}[systèmes I]

\textcolor{white}{s}

Résoudre les systèmes d’équations suivants.\\


$
\begin{tabular}{lll}
$(S_1)\left\lbrace
\begin{array}{lc}
a-b =0&\\
7b+3a =5&
\end{array}\right.$& 
\end{tabular}
$

\vspace{5cm}





$
\begin{tabular}{lll}
$(S_1)\left\lbrace
\begin{array}{lc}
a-b =0&\\
7b+3a =5&
\end{array}\right.$& $(S_2)\left\lbrace
\begin{array}{lc}
2a-4b =6&\\
3a-7b =8&
\end{array}\right.$ & $(S_3)\left\lbrace
\begin{array}{lc}
2a-b =12&\\
12a-6b=0&
\end{array}\right.$
% \\
%   2.1 & 2.2 & 2.3 \\
\end{tabular}
$
\end{exo}
\vspace{0.25cm}


\begin{exo}[systèmes II]

\textcolor{white}{s}

Résoudre les systèmes suivants puis vérifier graphiquement les résultats obtenus.\\

$
\begin{tabular}{ll}
$(S_1)\left\lbrace
\begin{array}{lc}
b-7a =4&\\
6a-3b =3&
\end{array}\right.$& $(S_2)\left\lbrace
\begin{array}{lc}
5b-3a =-1&\\
b+a =3&
\end{array}\right.$ 
\end{tabular}
$

\end{exo}
\vspace{0.25cm}



\begin{exo}[systèmes III]

\textcolor{white}{s}

Résoudre les problèmes suivants :
\vspace{0.25cm}

{\bfseries a)} aurélie dépense 5,80 euros pour six croissants et deux brioches. Il lui faudrait 0,40 euros de plus pour acheter deux croissants et six brioches.
Combien coûte chaque gâteau ?
\vspace{0.25cm}

{\bfseries b)} la salle compte 400 places. Les parterres sont à 23 euros et les balcons à 18 euros. Quand le théâtre est plein, la recette est de 8100 euros.
Combien y a-t-il de parterres, de balcons ?
\vspace{0.25cm}

{\bfseries c)} Déterminer deux entiers sachant que leur somme est 666 et que si on divise le plus grand par le plus petit le quotient est 3 et le reste 62.
\vspace{0.25cm}

{\bfseries d)} La différence de deux nombres est 24. Si l’on ajoute 8 à chacun de ces deux entiers, on obtient deux nouveaux entiers dont le plus grand est le triple du plus petit.
Quels sont ces entiers naturels ?
\vspace{0.25cm}

{\bfseries e)} un terrain rectangulaire a 220m de périmètre. En diminuant sa longueur de 2m et en augmentant sa largeur de 2m, son aire augmente de 16 m2.
Quelles sont les dimensions initiales du terrain ?
\end{exo}
\vspace{0.25cm}




\begin{exo}[systèmes IV]

\textcolor{white}{s}

{\bfseries a)} Trouver dans $\R$, les nombres x qui vérifient simultanément les deux inéquations données (on parle de systèmes d’inéquations) et représenter l’ensemble de solutions sur une droite graduée.\\


$
\begin{tabular}{l}
$(S_1)\left\lbrace
\begin{array}{lc}
2a-5 \geq-a+4&\\
3a+7 > 5a-6 &
\end{array}\right.$
\end{tabular}\\
$

{\bfseries b)} Résoudre les systèmes d’inéquations à deux inconnues.\\

$
\begin{tabular}{ll}
$(S_1)\left\lbrace
\begin{array}{lc}
3a-5b <8&\\
2a+3b >-3&
\end{array}\right.$& $(S_2)\left\lbrace
\begin{array}{lc}
2a+7b \geq 10&\\
-a+b <2&
\end{array}\right.$ 
\end{tabular}
$


\end{exo}
\vspace{0.25cm}


%5) Résoudre les systèmes d’inéquations à deux inconnues







\end{document}





